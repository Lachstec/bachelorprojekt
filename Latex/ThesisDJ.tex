\documentclass[12pt,oneside]{article}

%%%%%%%%%%%%%%%%%%%%%%%%%%%%
%%   Zusaetzliche Pakete  %%
%%%%%%%%%%%%%%%%%%%%%%%%%%%%
\usepackage{acronym}
\usepackage{enumerate}
\usepackage{a4wide}
\usepackage{fancyhdr}
\usepackage{graphicx}
\usepackage{palatino}
\usepackage{blindtext}
\usepackage{multirow}
\usepackage[section]{placeins}
\usepackage[ruled,longend]{algorithm2e}
\usepackage{float}
\usepackage{amsmath}
\usepackage{makecell}
\usepackage{tabularx}
\usepackage{amssymb}
\usepackage{subfiles}


%folgende Zeile auskommentieren für englische Arbeitenhttps://www.overleaf.com/project/62ecf4bc5c4c340d9e1ba297
\usepackage[ngerman]{babel}

\usepackage[bookmarks]{hyperref}
\usepackage[T1]{fontenc}
\usepackage[utf8]{inputenc}
\usepackage[a-1b]{pdfx}
\usepackage[justification=centering]{caption}
%\usepackage[style=unsrt,natbib=true,backend=biber]{biblatex}
\usepackage{csquotes}
\usepackage{url}

%%%%%%%%%%%%%%%%%%%%%%%%%%%%%%
%% Definition der Kopfzeile %%
%%%%%%%%%%%%%%%%%%%%%%%%%%%%%%

\pagestyle{fancy}
\fancyhf{}
\cfoot{\thepage}
\setlength{\headheight}{16pt}

%%%%%%%%%%%%%%%%%%%%%%%%%%%%%%%%%%%%%%%%%%%%%%%%%%%%%
%%  Definition des Deckblattes und der Titelseite  %%
%%%%%%%%%%%%%%%%%%%%%%%%%%%%%%%%%%%%%%%%%%%%%%%%%%%%%

\newcommand{\HSFTitle}[8]{

  \thispagestyle{empty}
\begin{center}
    \includegraphics[width=0.8\textwidth]{logo.eps} \\
    \vspace*{\stretch{1}}
    \end{center}

  %\vspace*{\stretch{1}}
  {\parindent0cm
  \rule{\linewidth}{.7ex}}
  \begin{center}
    \vspace*{\stretch{1}}
    \sffamily\bfseries\Huge
    #1\\
    \vspace*{\stretch{1}}
    \sffamily\bfseries\large
    #3
    \vspace*{\stretch{1}}
  \end{center}
  \rule{\linewidth}{.7ex}

  \vspace*{\stretch{2}}
  \begin{center}
    \Large #2 am #5 der HAW Fulda \\
    \vspace*{\stretch{1}}

    \large Matrikelnummer:  #4 \\[1mm]
    \large Dozent:  #7 \\[1mm]
    \large Fachbetreuer:  #8 \\[1mm]

    \vspace*{\stretch{1}}
    \large Eingereicht am #6
  \end{center}
}


%%%%%%%%%%%%%%%%%%%%%%%%%%%%
%%  Beginn des Dokuments  %%
%%%%%%%%%%%%%%%%%%%%%%%%%%%%



\begin{document}
\makeatletter
  \HSFTitle
      {Projektmanagement: Maßnahmen zur Optimierung der Nachrichtenverteilung in Gruppen und Organisationseinheiten }        % Titel der Arbeit
      {Hausarbeit} % Typ der Arbeit
      {Nicolas Will, Julian Dreuth, Leon Lux}          % Vor- und Nachname des Autors
      {?, ?, 1420067}
      {Fachbereich AI}  % Name des FBs
      {12.11.2024}        % Tag der Abgabe
      {Frank Breitenbach}     % Name des Erstgutachters
      {Daniel Brenzel}    % Name des Zweitgutachters
  \clearpage

\lhead{}
\pagenumbering{Roman}
    \setcounter{page}{1}

%%%%%%%%%%%%%%%%%%%%%%%%%%%%
%%  Kurzzusammenfassung   %%
%%%%%%%%%%%%%%%%%%%%%%%%%%%%
\clearpage
%
\markboth{Abstract}{Abstract}
\section*{Abstract}

Die folgende Arbeit beschäftigt sich mit der Anwendung von im Modul AI1021 erlernten Techniken des Projektmanagements anhand eines Projektes im 
Bereich J3 der Hessischen Zentrale für Datenverarbeitung (HZD). Ziel ist die Analyse und Optimierung der bereichsinternen Kommunikation mit dem Ziel, 
eine neue Messenging-Lösung zu finden, die bisher verwendete Lösungen ersetzen kann. Die schriftliche Ausarbeitung legt dabei ihren Fokus auf 
das Projektmanagement und Projektcontrolling. Dazu werden im Folgenden die durchlaufenen Phasen des Projektes dokumentiert und evaluiert, um als 
abschließenden Schritt gewonnene Erkenntnisse aufzuzeigen sowie ein Fazit zu ziehen.

\markboth{Sperrvermerk}{Sperrvermerk}
\section*{Sperrvermerk}
Die vorliegende Arbeit enthält Interna der HZD und ist daher streng vertraulich zu behandeln. Sie ist ausschließlich als Prüfungsvorlage bestimmt und weder
fremden Dritten zugänglich gemacht, noch ohne Einverständnisse der HZD vervielfältigt werden.


\clearpage
\tableofcontents
\clearpage

\addcontentsline{toc}{section}{\listfigurename}
\listoffigures

\addcontentsline{toc}{section}{\listtablename}
\listoftables
\clearpage


%%%%%%%%%%%%%%%%%%%%%%%%%%%%
%%  Einstellungen  %%
%%%%%%%%%%%%%%%%%%%%%%%%%%%%
\cleardoublepage
\pagenumbering{arabic}
    \setcounter{page}{1}
\lhead{\nouppercase{\leftmark}}

%%%%%%%%%%%%%%%%%%%%%%%%%%%%
%%  Hauptteil  %%
%%%%%%%%%%%%%%%%%%%%%%%%%%%%
\section{Einleitung}
Software spielt eine zunehmend zentrale Rolle als essenzieller Bestandteil und Innovationstreiber in Industrie, Verwaltung und Gesellschaft. Die Hessische Zentrale für Datenverarbeitung (HZD) entwickelt als zentraler IT-Dienstleister des Landes Hessen individuelle Softwarelösungen und führt Projekte in diesem Bereich durch. Ein wesentlicher Erfolgsfaktor bei der Realisierung solcher Projekte ist eine effiziente und strukturierte Kommunikation \cite{pikkarainen_impact_2008}. 

Derzeit nutzt die HZD eine Kombination aus E-Mails und der Messaging-Lösung Skype for Business, um in Projekten mit Mitarbeitenden sowie internen und externen Stakeholdern zu kommunizieren. Diese Lösung wird jedoch als unübersichtlich und ineffizient wahrgenommen. Aus diesem Grund plant die HZD die Einführung einer neuen Kommunikationsanwendung, die die bisherigen Werkzeuge ablöst. Ziel ist es, eine Plattform zu implementieren, die eine ergonomische Abbildung der Kommunikations- und Projektabläufe ermöglicht und so die Effizienz steigert.

Die vorliegende Arbeit untersucht das Projektmanagement dieses Vorhabens. Im Fokus stehen die eingesetzten Methoden und Prozesse im Kontext modernen Projektmanagements. Dazu wird das Projekt über alle Phasen hinweg begleitet, analysiert und kritisch reflektiert, um ein abschließendes Fazit ziehen zu können.


\section{Projektinitiierung}
\subfile{projektinitieerung.tex}

\section{Planungsphase}
In diesem Abschnitt wird die Planung des Projektes anhand der Aspekte Organisation, Kommunikation, Ressourcen, Risiken und Chancen, sowie Zeit erläutert.
Durch das Aufzeigen der Projektplanung kann in späteren Abschnitten aufgezeigt werden, inwiefern Abweichungen oder Verbesserungen im Vergleich 
zu der Planung in der Durchführungsphase aufgetreten sind.

\subsection{Organisation}

\subsection{Kommunikation}

\subsection{Ressourcen}

\subsection{Risiken und Chancen}

\subsection{Zeit}
Das Projekt findet in einem fest abgesteckten Zeitraum statt und ist mit einem festen Start- und Enddatum versehen.
Um die vom Kunden gewünschten Lieferobjekte fristgerecht fertigzustellen, wurde sich für eine Planung mithilfe von Meilensteinen mit
Fälligkeitsterminen entschieden. In Kollaboration im Projektteam wurde sich im Zuge eines Planungsmeetings auf folgende Meilensteine\ref{tab:milestones} verständigt: \\

\begin{centering}
\begin{tabular}{|c|c|c|c|c|}
  \hline
  Nummer & Code & Meilsteinname & Basis & Plan \\
  \hline
  1 & MS10 & Projektstart & 14.10.2024 & 14.10.2024 \\
  \hline
  2 & MS20 & Analyse des Ist-Zustandes abgeschlossen & 25.10.2024 & 25.10.2024 \\
  \hline
  3 & MS30 & Anforderungskatalog an Lösung festgelegt & 30.10.2024 & 30.10.2024 \\
  \hline
  4 & MS40 & Auswahl an Produkten zur Ablösung & 06.11.2024 & 06.11.2024 \\
  \hline
  5 & MS50 & Produkte wurden bewertet anhand des Katalogs & 22.11.2024 & 22.11.2024 \\
  \hline
  6 & MS60 & Produktentscheidung steht & 26.11.2024 & 26.11.2024 \\
  \hline
  7 & MS70 & Testumgebung beantragt & 28.11.2024 & 28.11.2024 \\
  \hline
  8 & MS80 & Proof-of-Concept aufgesetzt & 13.12.2024 & 13.12.2024 \\
  \hline
  9 & MS90 & Dokumentation angefertigt & 14.01.2025 & 14.01.2025 \\
  \hline
  10 & MS100 & Ergebnisse dem Kunden vorgestellt & 24.01.2025 & 24.01.2025 \\
  \hline
  11 & MS110 & Projekt-Retrospektive & 28.01.2025 & 28.01.2025 \\
  \hline
  12 & MS120 & Ende des Projektes & 31.01.2025 & 31.01.2025 \\
  \hline
\end{tabular}
  \caption Meileinsteine des Projekts
  \label{tab:milestones}
\end{centering}

\section{Projektdurchführung und Controlling}

\section{Wechsel der Kollaborationsplatform}
Im Rahmen eines unerwarteten Ereignisses trat ein zuvor nicht bekanntes Risiko auf: Die Kollaborationsplattform GitLab der Hochschule Fulda wurde zunächst in ihrer Verfügbarkeit eingeschränkt und anschließend vollständig deaktiviert. Die genauen Ursachen hierfür waren außerhalb der direkten Steuerungsmöglichkeiten, was eine akute Notwendigkeit zur Intervention nach sich zog. 
Da die Plattform eine zentrale Rolle für die Zusammenarbeit innerhalb des laufenden Projekts spielt, wurde ein außerordentliches Meeting einberufen, um geeignete Alternativen zu evaluieren. Im Verlauf der Analyse kristallisierten sich drei mögliche Optionen heraus:

\begin{enumerate}
	\item GitHub
	\item Azure DevOps (HZD-interne Anwendung)
	\item Gitlab Cloud
\end{enumerate}

Obwohl eine naheliegende Option darin bestand, auf die cloudbasierte Variante von GitLab zu wechseln, erwies sich dies als nicht realisierbar. Die damit verbundenen Kosten überschritten die verfügbaren finanziellen Mittel des Projekts, wodurch diese Option ausgeschlossen wurde. \\

Die verbleibenden Alternativen, Azure DevOps und GitHub, wurden daraufhin eingehend hinsichtlich ihrer Eignung für die Projekterfordernisse untersucht:

\begin{itemize}
	\item Azure DevOps bietet umfassende Funktionalitäten für die Zusammenarbeit. Jedoch beschränkt sich der Zugriff auf das interne Netzwerk der HZD, was die Zusammenarbeit außerhalb der festen Präsenztage erheblich einschränkt.
	\item GitHub stellt hingegen eine flexible und kostenfreie Lösung dar, die keine netzwerkspezifischen Einschränkungen aufweist und somit den Projektbeteiligten eine problemlose Nutzung ermöglicht.
\end{itemize}

Aufgrund dieser Faktoren fiel die Entscheidung zugunsten von GitHub. Die Migration der bestehenden Daten und Projekte konnte zügig und ohne nennenswerte Schwierigkeiten durchgeführt werden. Damit wurde das zuvor identifizierte Risiko erfolgreich gemindert und die Kontinuität der Projektarbeit sichergestellt.
Diese Vorgehensweise zeigt, wie wichtig eine proaktive Risikobewertung und schnelle Entscheidungsfindung im Projektmanagement sind, insbesondere bei der Nutzung essenzieller digitaler Infrastrukturen.


\section{Projektabschluss}



%%%%%%%%%%%%%%%%%%%%%%%%%%%%
%% Literaturverzeichnis wird
%% automatisch eingefügt
%%%%%%%%%%%%%%%%%%%%%%%%%%%%
\clearpage
\bibliographystyle{unsrt}
\bibliography{refs}

%\lhead{}
%\printbibliography
%\addcontentsline{toc}{section}{\bibname}
\appendix


\end{document}
