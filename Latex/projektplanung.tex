\documentclass[ThesisDJ.tex]{subfiles}

\begin{document}

In diesem Kapitel geht es um die Plannung des Projekts.Die Planungsphase gehört zu den wichtigsten Phasen des Projektmanagements. Hier werden finanzielle Mittel, Zeitplan und das vorhandene Personal zusammengefasst und der Projektplan erstellt. An diesem orientieren sich die Beteiligten während des gesamten Projekts. Was die Relevanz dieser Phase unterstreicht. Die Planungsphase wird dazu genutzt, um die Strukturen im Projekt aufzubauen. Dazu zählt das Verteilen von Rollen und Verantwortungen im Projekt. Weiterhin müssen kommunikationswege implementiert und Eskalationswege definiert werden. 
Zudem werden Meilensteine benannt und festgelegt, sowie die Arbeitspakete erstellt, um die Meilensteine zu erreichen. Eine ausführliche Risiko-Analyse ist auch Teil dieser Phase. 

\subsection{Arbeitspakete und Meilensteine}
Arbeitspakete teilen die Arbeit im Projekt in kleinere, überschaubare Aufgaben ein. Sie stellen die Kleinste Einheit im Projektstrukturplan dar und helfen dabei den Zustand des Projekts zu bewerten.
Ein Arbeitspaket hat ein klares Ziel, sowie definierte Start- und Endtermine, sowie einen defnierten Arbeitsumfang. Zusätzlich können Arbeitspakete in Abhängigkeit von anderen Paketen definiert werden und benötigte Ressourcen sowie eine Definition entahlten ab wann ein Paket als abgeschlossen gilt. Auch Verantwortlichkeiten können zugeornet werden. 
Arbeitspakete können dabei verschiedene Zustände haben. Eine Möglichkeit ist es den Paketen einen Diskreten Zustand zuzuordnen. Typischerweise wird unterschieden zwischen nicht bearbeitet, in Bearbeitung und abgeschlossen. Es kann jedoch auch Sinnvoll sein den Bearbeitungszustand in Prozent anzugeben. Dies setzt allerdings eine sinnvolle Metrik voraus. 

Bestimmt wurden die Arbeitspakete von den Studenten. Dabei haben alle Studenten unabhängig Arbeitspakete formuliert. Diese wurden dann zusammengetragen und ergänzt oder zusammengefasst.
Die Arbeitsaufwandschätzung wurde mit der Mittelwertmethode ermittelt. Dabei schätzen die Studenten den Aufwand für ein Arbeitspaket. Aus diesen Werten wird dann der Mittelwert bestimmt. Dieser wird dann als Aufwand für das Arbeitspaket angenommen. 
Da das Team sehr unerfahren mit dem Planen von Projekten ist, eignet sich diese Methode besonders gut, wenn man davon ausgeht dass die Schätzungen um den reellen Aufwand herum liegen. Durch den Mittelwert lassen sich nun abweichungen schmälern. Dabei werden natürlich bessere Ergebnisse erziehlt, je mehr Personen mitschätzen. 

Meilensteine in einem Projekt markieren den Abschluss eines bestimmten Ziels. Meistens werden diese auf ein bestimmtes Datum festgelegt. Sie dienen Dazu den Überblick in einem Projekt zu beahlten und den Verlauf des Projekts besser einschätzen zu können. Außerdem können diese als Motivator dienen, wenn ein gesetztes Ziel eingehalten werden kann. 
Auch die Meilensteine wurden zusammen festgelegt. Gewählt wurden bei diesem Projekt  


\subsection{Organisation und Kommunikation}
Das Projekt wird von 3 Dual Studirenden bearbeitet. Als Ansprechpartner dient ein ehemals Dual Studierender des gleichen Unternehmens. Die Dual Studierenden bearbeiten das Projekt wöchentlich an ihrem Praxistag. Zudem wird wöchentlich ein Meeting abgehalten, welches dazu dient, sich gegenseitig auf den aktuellen Stand zu bringen und eventuelle Anpassungen an der Planung vorzunehmen. Weitere Absprechungen mit Abteilungs- und Bereichsleitern werden nach Bedarf organisiert. In der Regel richten sich diese nach den Meilensteinen des Projekts.  

\subsection{Ressourcen}

Die Ressourcen umfassen die Arbeitskraft, die diesem Projekt zur Verfügung steht, sowie Räumlichkeiten. 
Die verfügbare Arbeitskraft berechnet sich aus der Anzahl der Mitarbeiter mulitipliziert mit den jeweiligen Personenstunden.
In diesem Projekt sind drei Dual Studierende welche jeweils eine Arbeitskraft von 8 Personenstunden pro Woche in das Projekt einbringen. 
Die Resultiernede Arbeitskraft entspricht damit:

		8 Personenstunden x 15 Tage x 3 = 360 Personenstunden

Weiterhin wurde den Studierenden ein Raum vor Ort zur Verfügung gestellt. Diese Räume beinhalten benötigte Medien, um ein gemeinsames arbeiten zu ermöglichen.

\subsection{Risiken und Chancen}
Ein weiteres wichtiges Element der Projektplanung ist die Risikoanalyse. Dabei werden Risiken, die das Projekt behindern können analysiert,
bewertet und ggf. Lösungen gesucht, wie auf das Risiko reagiert werden kann, falls es auftritt. 
Aber bei der Risikoanalyse werden auch Chancen betrachtet. Also Szenarien, die beim Eintreten positive Wirkungen auf das Projekt haben. 

\subsection{Zeit}
Das Projekt findet in einem fest abgesteckten Zeitraum statt und ist mit einem festen Start- und Enddatum versehen.
Um die vom Kunden gewünschten Lieferobjekte fristgerecht fertigzustellen, wurde sich für eine Planung mithilfe von Meilensteinen mit
Fälligkeitsterminen entschieden. Die Meilensteine wurden daraufhin in einem Planungsmeeting mit allen Mitgliedern des Projektes gefasst und festgehalten.
Eine Aufstellung der Meilensteine ist in Tabelle \ref{tab:milestones} zu finden.\\

Das Projekt ist auf den Zeitraum des Wintersemesters 24/25 an der Hochschule Fulda beschränkt.
Beginn ist der erste Praxistag der Studierenden ende ist der Letzte Praxistag der Studierenden. 
Damit Beginnt das Projekt am 15.10.2024 und endet am 04.03.2025. Es umfasst einen Zeitraum von 15 Wochen. 
Die Angaben in diesem Bericht beziehen sich also auf den Gesamten Projektverlauf.

\subsection{Stakeholdermanagement}

Während des Projekts ist ein gutes und kontinuierliches Stakeholdermanagement unerlässlich. 
Bei diesem geht es darum, Personen, Gruppen oder Organisationen zu identifizieren, die durch das Projekt in irgendeiner Weise betroffen sind oder das
Projekt in irgendeiner Weise betreffen und diese in das Projekt mit einzubinden. Wichtig dabei ist es zu analysieren, welche Position 
diese Stakeholder zu dem Projekt haben. Diese können in einer Stakeholdermatrix festgehalten werden (Siehe ...). 
Je früher mit dem Stakeholdermanagement begonnen werden kann desto besser.

Dabei kann folgender Prozess durchlaufen werden \cite[S.~487]{project_management_institute_guide_2017}:

Stakeholder identifizieren

Stakeholder zu identifizieren ist kein einmaliger Prozess. Im Projektverlauf können sich Stakeholder ändern. 
Je nach Phase können neue hinzukommen oder wegfallen. Auch die Einstellung zum Projekt kann sich im Projektverlauf ändern. 
Um auf die Auswirkungen der Stakeholder zu jederzeit reagieren zu können darf die Identifikation und Analyse der Stakeholder 
zu keiner Zeit vernachlässigt werden.

\subsubsection{Stakeholder Interaktion planen}
Stakeholder Interaktion planen

In diesem Schritt wird geplant wie der Kontakt zu Stakeholdern hergestellt werden kann und diese in das Projekt mit eingebunden werden können. 
Die Planung kann sich dabei an Kategorien wie Einfluss, Postition zum Projekt, Erwartungen oder Interessen orientieren.

\subsubsection{Stakeholder Interaktion Durchführen}


Im nächsten Schritt wird nun die Kommunikation und Zusammenarbeit mit den Stakeholdern begonnen. 
Dabei sollte über die Erwartungen und Probleme gesprochen werden.

\subsubsection{Stakeholder Interaktion Überwachen}

Nach den Gesprächen mit den Stakeholdern müssen diese ausgewertet werden, um die Einschätzung der Stakeholder im Bezug auf Einfluss,
Position zum Projekt oder Erwartungen und Interessen zu verbessern.


Der Prozess des Stakeholdermanagements ist iterativ angelegt. Die Dokumentation der Stakeholder sollte regelmäßig während
des Projekts stattfinden. Forschungen zum Scheitern von Projekten zeigen, dass die Relevanz des strukturierten vorgehens 
beim Stakeholdermanagements \cite[S.~488]{project_management_institute_guide_2017}.
Ein Besonderes Augenmerk muss auf das Stakeholdermanagement in folgenden Situationen gelegt werden:

\subsubsection{Das Projekt erreicht eine neue Phase}
Aktuelle Stakeholder sind nicht länger mit dem Projekt verbunden oder neue kommen hinzu
Es gibt signifikante Änderungen in der Organisation des Projekts oder bei Stakeholdern


\begin{table}[h]
  \centering
\begin{tabular}{|c|c|c|c|c|}
  \hline
  Nummer & Code & Meilsteinname & Basis & Plan \\
  \hline
  1 & MS10 & Projektstart & 14.10.2024 & 14.10.2024 \\
  \hline
  2 & MS20 & Analyse des Ist-Zustandes abgeschlossen & 25.10.2024 & 25.10.2024 \\
  \hline
  3 & MS30 & Anforderungskatalog an Lösung festgelegt & 30.10.2024 & 30.10.2024 \\
  \hline
  4 & MS40 & Auswahl an Produkten zur Ablösung & 06.11.2024 & 06.11.2024 \\
  \hline
  5 & MS50 & Produkte wurden bewertet anhand des Katalogs & 22.11.2024 & 22.11.2024 \\
  \hline
  6 & MS60 & Produktentscheidung steht & 26.11.2024 & 26.11.2024 \\
  \hline
  7 & MS70 & Testumgebung beantragt & 28.11.2024 & 28.11.2024 \\
  \hline
  8 & MS80 & Proof-of-Concept aufgesetzt & 13.12.2024 & 13.12.2024 \\
  \hline
  9 & MS90 & Dokumentation angefertigt & 14.01.2025 & 14.01.2025 \\
  \hline
  10 & MS100 & Ergebnisse dem Kunden vorgestellt & 24.01.2025 & 24.01.2025 \\
  \hline
  11 & MS110 & Projekt-Retrospektive & 28.01.2025 & 28.01.2025 \\
  \hline
  12 & MS120 & Ende des Projektes & 31.01.2025 & 31.01.2025 \\
  \hline
\end{tabular}

\caption{Meileinsteine des Projekts}
\label{tab:milestones}

\end{table}

\end{document}
