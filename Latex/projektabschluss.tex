%\documentclass[ThesisDJ.tex]{subfiles}
\documentclass{article}
\begin{document}
	\section{Veränderungen im Projektabschluss}
	Durch das Eintreten eines in der Planung nicht berücksichtigten Risikos für das Projekt musste der Projektabschluss in mehreren Punkten an die veränderte Situation angepasst werden. Dies betrifft sowohl das bzw. die Lieferobjekte als auch die damit zusammenhängende Übergabe und Anwendung der Resultate.\\
	Die Einführung der Kollaborationsplattform WebEx von Cisco hat dazu geführt, dass sich die Rahmenbedingungen für die Wahl eines neuen Systems zur Kollaboration im Bereich J3 insofern verändert haben, als dass zunächst beurteilt werden musste, ob mit dieser neuen Plattform überhaupt noch ein zusätzliches System zur Kommunikation im Bereich J3 notwendig ist. \\
	Somit verändert sich das Lieferobjekt dergestalt, dass es sich entweder um die ursprünglich festgelegten Lieferobjekte handelt oder um eine Information, dass Cisco WebEx schon so viele oder gar alle Anforderungen an die neue Software erfüllt und somit eine Empfehlung abgegeben werden kann, dass kein neues Tool zur Kommunikation eingeführt werden sollte.\\
	
	\section{Ergebnis des Projekts}
	Da das Resultat der Evaluation war, dass keine zusätzliche Software eingeführt werden sollte, verkürzt sich das Projekt dahingehend, dass für WebEx keine Konfiguration ermittelt werden muss, die für die Anwendung im Bereich J3 funktioniert, da der HZD-weite Rollout von WebEx bereits die zu nutzende Konfiguration vorgibt. Der eigentliche Projektabschluss beschränkt sich damit auf ein Dokument, das die Evaluation der einzelnen Kollaborationsplattformen dokumentiert und die Entscheidung, keine weitere Software zusätzlich zu WebEx einzuführen, transparent nachvollziehbar macht. \\
	Dieses Ergebnis ist Resultat aus der Evaluation der Kollaborationsplattformen mit dem Ergebnis, dass WebEx bereits die geforderten Funktionen abdeckt und somit kein weiteres Produkt notwendig ist. Die Evaluation umfasste dabei eine detaillierte Analyse der Funktionalitäten von WebEx im Vergleich zu anderen verfügbaren Plattformen. Dabei wurden insbesondere Aspekte wie Benutzerfreundlichkeit, Integration in bestehende Systeme, Sicherheitsstandards und Kosten berücksichtigt. Die Ergebnisse dieser Analyse wurden in einem umfassenden Bericht zusammengefasst, der die Entscheidungsgrundlage für den Verzicht auf eine zusätzliche Software bildete. \\
	
	\section{Projektübergabe}
	Das Ergebnis des Projekts wurde im Rahmen einer kurzen Präsentation den Auftraggebern aus dem Bereich J3 vorgestellt, in dessen Rahmen auch die Dokumentation der Evaluationen der einzelnen Software-Produkte übergeben wurde. Da sich aus dem Ergebnis keine Veränderungen für die Arbeit des Bereichs J3 ergeben, war kein umfangreicherer Projektabschluss notwendig, in dem beispielsweise noch auf den Umgang mit der neuen Plattform eingegangen werden muss oder Fragen zur Nutzung der neuen Software beantwortet werden müssen. Für diese Fragen ist durch die zentrale Einführung von WebEx Cisco ein anderer Bereich der HZD zuständig, an den entsprechende Fragen zu richten sind. \\
	Die Präsentation umfasste eine Zusammenfassung der wichtigsten Erkenntnisse aus der Evaluation, eine Darstellung der Entscheidungsfindung sowie eine Erläuterung der Konsequenzen für den Bereich J3. Dabei wurde insbesondere betont, dass durch die Nutzung von WebEx keine zusätzlichen Schulungen oder Anpassungen der Arbeitsprozesse notwendig sind, was eine erhebliche Zeit- und Kostenersparnis für den Bereich J3 bedeutet. Die Dokumentation der Evaluation wurde den Auftraggebern in digitaler Form übergeben, um eine einfache Nachvollziehbarkeit und Archivierung zu gewährleisten. \\
	
	\section{Reflexion und Lessons Learned}
	Die Erfahrungen aus diesem Projekt haben gezeigt, wie wichtig eine flexible Anpassung der Projektplanung an veränderte Rahmenbedingungen ist. Das unerwartete Eintreten des Risikos, das zur Einführung von WebEx führte, erforderte eine schnelle und effiziente Reaktion des Projektteams. Dabei wurde deutlich, dass eine enge Zusammenarbeit mit den Stakeholdern und eine transparente Kommunikation entscheidend für den erfolgreichen Abschluss des Projekts waren. \\
	Ein weiterer wichtiger Aspekt war die Bedeutung einer gründlichen und objektiven Evaluation der verfügbaren Lösungen. Durch die detaillierte Analyse der verschiedenen Kollaborationsplattformen konnte sichergestellt werden, dass die getroffene Entscheidung auf einer soliden Grundlage beruhte und alle relevanten Faktoren berücksichtigt wurden. Diese Herangehensweise hat nicht nur die Akzeptanz der Entscheidung bei den Auftraggebern erhöht, sondern auch das Vertrauen in die Kompetenz des Projektteams gestärkt. \\
	Abschließend lässt sich festhalten, dass das Projekt trotz der unvorhergesehenen Herausforderungen erfolgreich abgeschlossen werden konnte. Die gewonnenen Erkenntnisse und Erfahrungen werden zukünftige Projekte positiv beeinflussen und dazu beitragen, ähnliche Situationen noch effizienter zu bewältigen.
\end{document}