%\documentclass[ThesisDJ.tex]{subfiles}
\documentclass{article}
\begin{document}
	\section{Veränderungen im Projektabschluss}
	Durch das Eintreten eines in der Planung nicht berücksichtigten Risikos für das Projekt musste der Projektabschluss in meheren Punkten an die veränderte Situation angepasst werden. Dies betrifft sowohl das bzw. die Lieferobjekte als auch die damit zusammenhängende Übergabe und Anwendung der Resultate.\\
	Die Einführung der Kollaborationsplattform WebEx von Cisco hat dazu geführt, dass sich die Rahmenbedingungen für die Wahl eines neuen Systems zur Kollaboration im Bereich J3 insofern verändert haben, als dass zunächst beurteilt werden musste, ob mit dieser neuen Plattform überhaupt noch ein zusätzliches System zur Kommunikation im Bereich J3 notwendig ist. \\
	Somit verändert sich das Lieferobjekt dergestalt, dass es sich entweder um die ursprünglich festgelegten Lieferobjekte handelt oder um eine Information, dass Cisco WebEx schon so viele oder gar alle Anforderungen an die neue Software erfüllt und somit eine Empfehlung abgegeben werden kann, dass kein neues Tool zur Kommunikation eingeführ werden sollte.\\
	
	\section{Ergebnis des Projekts}
	Da das Resultat der Evaluation war, dass keine zusätzliche Software eingeführt werden sollte, verkürzt sich das Projekt dahingehend, dass für WebEx keine Konfiguration ermittelt werden muss, die für die Anwendung im Bereich J3 funktioniert, da der HZD-weite Rollout von WebEx bereits die zu nutzende Konfiguration vorgibt. Der eigentliche Projektabschluss beschränkt sich damit auf ein Dokument, das die Evaluation der einzelnen Kollaborationsplattformen dokumentiert und die Entscheidung, kein weitere Software zusätzlich zu WebEx einzuführen, transparent nachvollziehbar macht. \\
	Dieses Ergebnis ist Resultat aus der Evaluation der Kollaborationsplattformen mit dem Ergebnis, dass Webex bereits die geforderten Funktionen abdeckt und somit kein weiteres Produkt notwendig ist.
	
	\section{Projektübergabe}
	Das Ergbnis des Projekts wurde im Rahmen einer kurzen Präsentation den Auftraggebern aus dem Bereich J3 vorgestellt, in dessen Rahmen auch die Dokumentation der Evaluationen der einzelnen Software-Produkte übergeben wurde. Da sich aus dem Ergebnis keine Veränderungen für die Arbeit des Bereichs J3 ergeben, war kein umfangreicherer Projektabschluss notwendig, in dem beispielsweise noch auf den Umgang mit der neuen Plattform eingegangen werden muss oder Fragen zur Nutzung der neuen Software beantwortet werden müssen. Für diese Fragen ist durch die zentrale Einführung von WebEx Cisco ein anderer Bereich der HZD zuständig, an den entsprechende Fragen zu richten sind.
\end{document}