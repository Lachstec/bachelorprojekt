\documentclass[ThesisDJ.tex]{subfiles}

\begin{document}

Im weiteren Verlauf des Projekts wurde nachträglich die Anforderung formuliert, dass die zu evaluierende Software ebenfalls barrierefrei nutzbar sein muss. Dies bedeutet, dass die Anwendung den umfassenden Prinzipien der Web Content Accessibility Guidelines (WCAG)\cite{Kirkpatrick2018WCAG21} entsprechen muss und somit auch für Menschen mit unterschiedlichen Einschränkungen, beispielsweise im Bereich der Seh-, Hör- oder motorischen Fähigkeiten, problemlos zugänglich sein sollte.\\
Die rechtlichen Grundlagen, die ein Befolgen der Prinzipien der WCAG für einen Landesbetrieb wie die HZD verpflichtend machen finden sich dabei in der Barrierefreie-Informationstechnik-Verordnung 2.0 (BITV 2.0)\cite{bitv2}, dem Barrierefreiheitsstärkungsgesetz\cite{bfsg}, sowie mittelbar der DIN EN 301 849\cite{din301549}. Dabei definiert die DIN EN 301 549 Barrierefreiheitsanforderungen für Informations- und Kommunikationstechnologie -Produkte und -Dienstleistungen, womit die Prinzipien der genannten WCAG abgebildet sind.

\subsection{Theoretische Auswirkungen auf den Projektablauf}
Die nachträgliche Einbringung dieser zusätzlichen Anforderung hat weitreichende Auswirkungen auf den gesamten Projektverlauf und bedingt einige Anpassungen und Erweiterungen in verschiedenen Bereichen:

\begin{itemize}
    \item \textbf{Erweiterung der Evaluationskriterien:} Die ursprünglich festgelegten Bewertungskriterien müssen nun zwingend um zusätzliche Aspekte im Bereich der Barrierefreiheit ergänzt werden. Dazu gehört unter anderem die Berücksichtigung von ausreichend hohen Farbkontrasten, der durchgehenden Bedienbarkeit mit der Tastatur sowie der reibungslosen Kompatibilität mit Screenreader-Software, die von Menschen mit Sehbehinderungen genutzt wird.
    \item \textbf{Zusätzlicher Testaufwand:} Da die Einhaltung von Barrierefreiheitsstandards nicht immer unmittelbar ersichtlich ist, sind spezifische Tests erforderlich, um die Einhaltung der Anforderungen sicherzustellen. Diese Tests müssen entweder durch geschulte Testpersonen mit entsprechenden Bedürfnissen oder durch automatisierte Prüfwerkzeuge vorgenommen werden, was einen erhöhten zeitlichen und organisatorischen Aufwand bedeutet. 
    \item \textbf{Zeitliche Auswirkungen:} Durch die notwendige, zusätzliche Evaluierung sowie potenzielle Anpassungen an der Software könnte sich der ursprünglich festgelegte Zeitrahmen des Projekts verlängern. Falls das Projekt bereits einem engen Zeitplan unterliegt, müssten eventuell Meilensteine oder Deadlines angepasst oder neu definiert werden, um den zusätzlichen Anforderungen gerecht zu werden.
    \item \textbf{Mögliche Anpassungen an der Software:} Falls die geprüfte Software die neu hinzugekommenen Barrierefreiheitskriterien nicht in ausreichendem Maße erfüllt, könnte es erforderlich sein, alternative Softwarelösungen zu evaluieren oder in den Dialog mit den jeweiligen Entwicklern zu treten, um herauszufinden, ob und in welchem Umfang Anpassungen an der bestehenden Software vorgenommen werden können.
    \item \textbf{Zusätzliche Schulungen und Dokumentation:} Falls Barrierefreiheit bislang nicht als zentraler Bestandteil des Projekts betrachtet wurde, könnte es notwendig sein, das gesamte Projektteam in Bezug auf die relevanten Aspekte barrierefreier Softwareentwicklung und Evaluierung zu sensibilisieren. Darüber hinaus könnte zusätzliche Dokumentation erforderlich sein, um die Umsetzung und Nutzung barrierefreier Prinzipien nachvollziehbar darzustellen und langfristig zu sichern.
\end{itemize}

Zusammenfassend führt die nachträgliche Anforderung zur Einhaltung von Barrierefreiheitsstandards zu einer signifikanten Erweiterung des Evaluationsprozesses, zu möglichen Verzögerungen im Projektverlauf und zu einem erhöhten Bedarf an Ressourcen. Eine frühzeitige Berücksichtigung solcher Anforderungen bereits in der Planungsphase hätte potenzielle Umstellungen im Projektablauf minimieren und effizienter gestalten können.

\subsection{Praktischer Einfluss auf das Projekt}
Die Erweiterung des Anforderungskatalogs um das Kriterium der Barrierefreiheit der Software fiel in einen zeitlich eng mit der Ankündigung des Umstiegs der gesamten HZD auf die Software Webex der Firma Cisco korrelierenden Zeitrahmen. Da die Software nicht hätte eigeführt werden dürfen, ohne dass entsprechende Barrierefreiheitsbedingungen erfüllt sind, kann an dieser Stelle davon ausgegangen werden, dass eine entsprechende Prüfung durch den zuständigen Fachbereich positiv ausgefallen ist und Cisco Webex somit im Rahmen der genannten gesetzlichen und normativen Grundlagen barrierefrei ist. Ein erneute Prüfung innheralb des Projekts erscheint dann nicht mehr sinnvoll, einer Überprüfung durch einen anderen Fachbereich der HZD darf durchaus so viel Vertrauen entgegengebracht werden, dass eine entsprechende Einschätzung übernommen werden kann.\\
Somit ist für Webex keine weitere Überprüfung notwendig und durch die parallel durchgeführten Evaluationen, die den Einsatz einer anderen, zusätzlichen Software sehr unwahrscheinlich darstellten, konnten auch Evaluationen der anderen ursprünglich in Betracht gezogenen Produkte zumindest so lange ausgesetzt werden, bis sich die Situation dahingehend ändert, dass eines dieser Produkte doch wieder zur Nutzung in Betracht gezogen wird.

\end{document}
